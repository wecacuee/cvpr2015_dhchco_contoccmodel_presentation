With the understanding of our occlusion aware 3D representation of objects, we are know in position to describe the energies that we defined in Chapter~\ref{sec:setup}

%%%%%%%%%%%%%%%%%%%%%%%%%%%%%%%%%%%%%%%%%%%%%%%%%%%%
\section{Point tracks energy with occlusion}
\label{sec:totalContPtTracksEnergy}
We model continuous point tracks energy with explicit occlusion reasoning as
the expected re-projection error over the association probability,

\begin{align}
  \Energy{track}(\{ \relp{i}{t} \}_i, \{ \relp{i}{t-1} \}_i, \{\dimsn{i}\}_i ) = 
    \sum_{i=1}^{N} 
    %\sum_{t = s_i}^{e_i}
    \sum_{j = 1}^{M}
    \int_1^\infty \assocP\Ereproj(\lambda) d\lambda
\end{align}
where $\assocP$ is the association probability of
$j$\textsuperscript{th} point with $i$\textsuperscript{th} TP at depth $\lambda$, $\{ \relp{i}{t} \}_i$ are the poses of all occluding objects at time $t$, $ \{ \dimsn{i} \}_i$ are the dimensions of all objects that occlude $i$
and $\Ereproj(\lambda)$ is the re-projection error given by
%
\begin{align}
  \assocP &= \Prefl\Ptrans\\
  \Ereproj(\lambda) &= \left\|\trackpj{t} - \projectionOf{\invProjectionOftm{\trackpj{t-1}, \lambda}}\right\|^2 .
  \label{eq:reprojerror}
\end{align}

The  $\projectionOf{.}$ and $\invProjectionOftm{.}$ denote the projection and
inverse projection functions that project 3D point to camera image and vice
versa. Note that inverse projection $\invProjectionOf{.}$ depend on both the
point $\trackp{t}$ and the unknown depth $\lambda$. Also note that the inverse projection is dependent on TP pose at time $t-1$ while the projection depends on pose at time $t$ which can be different.

\begin{comment}
  \subsection{Occupancy function}
  Assuming occupancy to be a
  probability distribution over 3D space. For each TP the
  occupancy is modeled as a logistic function 
  \begin{align}
     \occf = L(\mathbf{x}; \pos{i}{t}, \Sigma_i)
  \end{align}
  where $L(.)$ is the logistic function defined by
  \begin{align}
    L(\mathbf{x}; \pos{i}{t-1}, \Sigma_i) = \frac{1}{
      1 + e^{-k(1 - d(\mathbf{x},\pos{i}{t-1}))}
      }
  \end{align}
  where $d(\mathbf{x},\pos{i}{t-1}) =
  (\mathbf{x}-\pos{i}{t-1})^\top\Sigma_i(\mathbf{x}-\pos{i}{t-1})$ and $k =
  10\ln{49}$. $k$ is chosen such that $L(.) = 0.98$ when $d(.) =
  0.9$
  % Once we have our distribution over $\lambda$, $\lambdadist$ we can compute the
  % reprojection of $\trackpj{t-1}$ over image in time $t$ as a function of
  % $\lambda$. Let
\end{comment}


\begin{comment}
  \subsection{Approximations}
  Reflection probability of $i$th TP is easy to compute
  analytically 
  \begin{multline}
    \Prefl =
    (\max \{ 0, \nabla \occf^\top \ray \})^2 \\
    = (\max \{ 0, \nabla \occfxi^\top\ray \})^2
    \label{eq:analytic-prefl}
  \end{multline}
  where 
  \begin{multline}
    \nabla \occfxi^\top\ray \\=
    \nabla k\dishort \sech^2\left(\frac{k}{2}\dishort\right)
  \end{multline}
  where $\dishort = 1-d(\mathbf{x}, \pos{i}{t-1})$ is a signed distance measure
  from the contour of the ellipsoid where $d(\mathbf{x}, \pos{i}{t-1})$ is 1.

  However, the transmission probability needs to be approximated.  
  %
  % \subsection{Computing $\Prefl$ and $\Ptrans$}
  % Focusing on  $\nabla \occfxi$ 
  % 
  % \begin{multline}
  %   \nabla \occfxi =\\
  %   \frac{-k\nabla d(\mathbf{x}, \pos{i}{t-1})e^{-k(1-d(\mathbf{x}, \pos{i}{t-1}))}}{
  %     (1 + e^{-k(1-d(\mathbf{x}, \pos{i}{t-1}))})^2
  %   } \\ 
  %   = -k\nabla d(\mathbf{x}, \pos{i}{t-1})\sech^2\left(\frac{k}{2}(1-d(\mathbf{x},
  %   \pos{i}{t-1}))\right)
  %   \\
  %   = \nabla k\dishort \sech^2\left(\frac{k}{2}\dishort\right)
  % \end{multline}
  % where $\dishort = 1-d(\mathbf{x}, \pos{i}{t-1})$ is a signed distance measure
  % from the contour of the ellipsoid where $d(\mathbf{x}, \pos{i}{t-1})$ is 1.
  % Focusing on $\nabla d(.)$
  % 
  % \begin{align}
  %   \nabla d(\mathbf{x}, \pos{i}{t-1}) = 2\Sigma_i(\mathbf{x} - \pos{i}{t-1})
  % \end{align}
  % Back to \eqref{eq:analytic-prefl}
  % 
  % \begin{multline}
  %   (\max \{ 0, \nabla \occf^\top \ray \})^2 = \\
  %   \sech^4\left(\frac{k}{2}\dishort\right) 
  %   (\max \{ 0 , \nabla k\dishort^\top\ray\})^2
  % \end{multline}
  % 
  % The probability is simply $\sech^2(.)$ scaled by gradient of ellipsoid
  % $\nabla k\dishort$ projected in the ray direction $\ray$.
  % 
  % \begin{align}
  %   \Ptrans = 
  %   e^{\int_{1}^{\lambda} \ln{(1 - f_{occ}(\lambda \ray))}{d\lambda}}
  % \end{align}
  % \begin{multline}
  % \int_{1}^{\lambda} \ln{(1 - f_{occ}(\lambda \ray))}{d\lambda}
  % =  \\
  % \int_{1}^{\lambda} \ln{\left(1 - \sum_i \occfi\right)}{d\lambda}
  % \end{multline}
  % 
  % The above integral is very difficult to approximate or compute analytically.
  So based on intuition, we approximate the $\Ptrans$ by following function
  \begin{align}
  \label{eq:evalCumulativePtrans}
    \Ptrans &= \prod_i L_u(\mathbf{u}, \mu^i_u,\Sigma^i_u)L_{\lambda}(\lambda; \mu^i_d)\\
    L_u(\mathbf{u}, \mu^i_u,\Sigma^i_u) &= \frac{1}{
      1 + e^{-k_u(1 - (\mathbf{u} - \mu^i_u)^\top\Sigma^i_u(\mathbf{u} -
      \mu^i_u))}
    }
    \\
    L_{\lambda}(\mathbf{u}, \lambda; \mu^i_d) &= \frac{1}{
    1 + e^{-k_d(\lambda - \mu^i_d(\mathbf{u}))}
  }
  \end{align}
  where 
  \begin{align}
    \mu_u^i &= \projectionOf{\pos{i}{t-1}} \label{eq:muiudef}\\
    \Sigma_u^i &= \projectionOf{\Sigma_i} \label{eq:sigmauidef}\\
    \mu^i_d(\mathbf{u}) &= \relp{i}{t}\\
    k_u &= 10\log(49)\\
    k_d &= \frac{\log(49)}{\sqrt{h^2 + l^2 + w^2}}
    \label{eq:ptransmissionInit}
  \end{align}
  is the distance of the centre of the TP from the camera.

  % $\mu^i_d(\mathbf{u}) = \min_{\lambda} d^2_i(\lambda K^{-1}\mathbf{u})$.
  % $\mu^i_d(\mathbf{u})$ is the closest point to the unit contour of ellipsoid.
  % If there are multiple such points, the point closest to the camera is taken as
  % $\mu^i_d(\mathbf{u})$

  The association probability becomes

  \begin{multline}
    \assocP = 
      \sech^4\left(\frac{k}{2}\dishort\right)
      (\max \{ 0, \nabla k\dishort^\top\ray \})^2\\
    \prod_i \Lu
      \Llambda \\
      \label{eq:assocCoeffEval}
  \end{multline}

  So the energy becomes

  \begin{multline}
    \label{eq:integrand}
    \Energy{track}(.) = 
      \sum_{i = 1}^N
      \sum_{j = 1}^{M}
      \int_1^{\infty}
      \assocP
      \Ereproj(\lambda)
      d\lambda
  \end{multline}
  where $x = \lambda \ray$ and $\Ereproj(\lambda) = \|\trackpj{t} -
  f^i_{reproj}(\trackpj{t-1}, \lambda)\|^2$ is reprojection error which is a
  quadratic in $\lambda$

  The integral in the above expression is computed numerically.
\end{comment}

\section{Object detection energy}

We model object detection energy as the norm of difference in the coordinates
of the projected bounding box and detected bounding box. The detected bounding
box is represented by the extrema along X and Y dimensions $\bb{i} = [\xmin,
\ymin, \xmax, \ymax]^\top$. If $\dimsn{i} = [l, w, h]^\top$ represent the 3D
dimensions of the traffic participant along the X, Y and Z dimensions then we 
can compute the eight corners of the cuboid model in the tracklet coordinate transform as

\begin{align}
  X^{(i)}_{\text{cuboid}} &= \dimsn{i}{}^\top C_u \\
  \text{where} &\\
  C_u &= 
  \begin{bmatrix}
  0.5 & 0.5 & -0.5 & -0.5 & 0.5 & 0.5 & -0.5 & -0.5 \\
    0.5 & -0.5 & -0.5 & 0.5 & 0.5 & -0.5 & -0.5 & 0.5 \\
    0 & 0 & 0 & 0 & 1  & 1 & 1 & 1
  \end{bmatrix} \enspace .
\end{align}
These cuboid points can be projected to the camera image plane to get the projected bounding box,
\newcommand{\ucub}{U^{(i)}(t)}
\newcommand{\ucubx}{U_{(1,:)}^{(i)}(t)}
\newcommand{\ucuby}{U_{(2,:)}^{(i)}(t)}
\newcommand{\estbb}{\hat{\mathbf{d}}^{(i)}(t)}
\begin{align}
  \ucub &= \projectionOf{\dimsn{i}{}^\top C_u}\\
  \estbb &= [\min \ucubx, \min \ucuby, \max \ucubx, \max \ucuby]^\top \enspace .
\end{align}
We take $\estbb$ as the projected bounding box. The object detection energy is
taken to be the euclidean distance between projected and detected bounding box.
\begin{align}
  \EnergyBBox &= \| \estbb - \bb{i}\|^2_2
\end{align}
With slight abuse of notation for projection operator we write this energy as
\begin{align}
  \EnergyBBox &= \left\| \projectionOf{\dimsn{i}} - \bb{i}\right\|^2_2 \enspace.
\end{align}

For object detections we use object detector by \cite{Felzenszwalb_etal_2010}
which is detector by parts model and we use eight parts to train the car model 
on half of the KITTI dataset \cite{geiger2013vision}. The trained modeled is 
used to get detections for the other half of the dataset and vice versa.

\section{Lane energy}
\label{sec:laneEnergy}
 The lanes are modeled as splines. Here we assume that the confidence in lane
 detection is decreases as the distance from the lane center increases.  The
 energy is given by the dot product between car orientation and tangent to the
 lane at that point.

\begin{align}
  \label{eq:laneOrientationEnergy}
  \Energy{lane} = 
  \sum_{m \in M_{\text{close}}}
  (1 - \ori{i}{t} \cdot \text{TAN}(L_{m}(k), \pos{i}{t}) )
\LaneUncertainty{\pos{i}{t}}
\end{align}
where $M_{\text{close}} = \{m : \text{DIST}(L_{m}(k), \pos{i}{t}) < d_{\text{thresh}}\} $ is
the set of nearby lanes close to the object by a certain distance threshold $d_{\text{thresh}}$ and 
\begin{align}
\LaneUncertainty{\pos{i}{t}} = 
  \frac{1}{1 + \exp(-q(w_{\text{road}} - \text{DIST}(L_{m}(k), \pos{i}{t})))}
\end{align}
for some constant $w_{\text{road}}$ that represents the width of the road.

\section{Transition probability}
Dynamics constraints should not only enforce smooth trajectories, but also the
holonomic constraints.  The following energy adds a penalty if the change in
position is not in direction of previous orientation.

\begin{align}
  \label{eq:totalPosTransitionEnergy}
  \Energy{dyn-hol} = 1 - \ori{i}{t-1} \cdot (\pos{i}{t} - \pos{i}{t-1})
\end{align}

The following energy adds a penalty for change in position and orientation
but a penalty for change in velocity is much better approximation. However, in
a Markovian setting that would mean extending the state space of the car to
include velocity.

\begin{align}
  \Energy{dyn-ori} &= \|\ori{i}{t} - \ori{i}{t-1}\|^2\\
  \Energy{dyn-vel} &= \|(\pos{i}{t} - 2\pos{i}{t-1}) + \pos{i}{t-2}\|^2
\end{align}

As a result the dynamics are modeled by weighted combination of holonomic
constraint and smoothness constraints.

\begin{align}
  \WEnergy{dyn} &= \WEnergy{dyn-hol} + \WEnergy{dyn-ori} + \WEnergy{dyn-vel}
\end{align}

 
\section{Collision energy}

Bhattacharya coefficient $\int_a^b\sqrt{p(x)q(x)}dx$ is a measure of similarity
of two distributions $p(x)$ and $q(x)$. If we represent TPs as gaussians in
Birds eye view (BEV), then similarity is a measure of collision. Exactly
overlapping distribution results in coefficient as 1. 
%The analytical form of
%Bhattacharya coefficient has been taken from
%\url{http://like.silk.to/studymemo/PropertiesOfMultivariateGaussianFunction.pdf}

\begin{align}
  \label{eq:collisionEnergyHellingerDistance}
  \EnergyCol =
  -\log\left(
  A_{ij}
  e^{-\frac{1}{8}
    \left(\pos{i}{t} - \pos{j}{t}\right)^\top
    P^{-1}
    \left(\pos{i}{t} - \pos{j}{t}\right)
    }
    \right)
\end{align}
where 
\begin{multline}
  A_{ij} = \frac{|\Sigma_i|^\frac{1}{4}|\Sigma_j|^\frac{1}{4}}{|P|^\frac{1}{2}}, \hfill
  P = \frac{1}{2}\Sigma_i + \frac{1}{2}\Sigma_j, \hfill
\Sigma_i^{-1} = R^\top_{\ori{i}{t}} \begin{bmatrix} 2/l_i & 0 \\ 0 & 2/w_i \end{bmatrix}
R_{\ori{i}{t}}
\end{multline}


\section{Size Prior}

Prior can include among many other things the size prior on the car.

\begin{align}
  \label{eq:totalSizeEnergy}
  \Energy{size} &= (\dimsn{i} - \expDimsn)^\top\Sigma_{\expDimsn}^{-1}(\dimsn{i} -
  \expDimsn)
\end{align}

where $\expDimsn$ is the mean TP dimensions and
$\Sigma_{\expDimsn}$ is the correspondence covariance matrix.
